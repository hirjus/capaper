\documentclass[finnish,]{book}
\usepackage{lmodern}
\usepackage{amssymb,amsmath}
\usepackage{ifxetex,ifluatex}
\usepackage{fixltx2e} % provides \textsubscript
\ifnum 0\ifxetex 1\fi\ifluatex 1\fi=0 % if pdftex
  \usepackage[T1]{fontenc}
  \usepackage[utf8]{inputenc}
\else % if luatex or xelatex
  \ifxetex
    \usepackage{mathspec}
  \else
    \usepackage{fontspec}
  \fi
  \defaultfontfeatures{Ligatures=TeX,Scale=MatchLowercase}
\fi
% use upquote if available, for straight quotes in verbatim environments
\IfFileExists{upquote.sty}{\usepackage{upquote}}{}
% use microtype if available
\IfFileExists{microtype.sty}{%
\usepackage{microtype}
\UseMicrotypeSet[protrusion]{basicmath} % disable protrusion for tt fonts
}{}
\usepackage[margin=1in]{geometry}
\usepackage{hyperref}
\hypersetup{unicode=true,
            pdftitle={Korrespondenssianalyysi - graafinen ja geometrinen data-analyysin menetelmä},
            pdfauthor={Jussi Hirvonen},
            pdfborder={0 0 0},
            breaklinks=true}
\urlstyle{same}  % don't use monospace font for urls
\ifnum 0\ifxetex 1\fi\ifluatex 1\fi=0 % if pdftex
  \usepackage[shorthands=off,main=finnish]{babel}
\else
  \usepackage{polyglossia}
  \setmainlanguage[]{finnish}
\fi
\usepackage{natbib}
\bibliographystyle{apalike}
\usepackage{longtable,booktabs}
\usepackage{graphicx,grffile}
\makeatletter
\def\maxwidth{\ifdim\Gin@nat@width>\linewidth\linewidth\else\Gin@nat@width\fi}
\def\maxheight{\ifdim\Gin@nat@height>\textheight\textheight\else\Gin@nat@height\fi}
\makeatother
% Scale images if necessary, so that they will not overflow the page
% margins by default, and it is still possible to overwrite the defaults
% using explicit options in \includegraphics[width, height, ...]{}
\setkeys{Gin}{width=\maxwidth,height=\maxheight,keepaspectratio}
\IfFileExists{parskip.sty}{%
\usepackage{parskip}
}{% else
\setlength{\parindent}{0pt}
\setlength{\parskip}{6pt plus 2pt minus 1pt}
}
\setlength{\emergencystretch}{3em}  % prevent overfull lines
\providecommand{\tightlist}{%
  \setlength{\itemsep}{0pt}\setlength{\parskip}{0pt}}
\setcounter{secnumdepth}{5}
% Redefines (sub)paragraphs to behave more like sections
\ifx\paragraph\undefined\else
\let\oldparagraph\paragraph
\renewcommand{\paragraph}[1]{\oldparagraph{#1}\mbox{}}
\fi
\ifx\subparagraph\undefined\else
\let\oldsubparagraph\subparagraph
\renewcommand{\subparagraph}[1]{\oldsubparagraph{#1}\mbox{}}
\fi

%%% Use protect on footnotes to avoid problems with footnotes in titles
\let\rmarkdownfootnote\footnote%
\def\footnote{\protect\rmarkdownfootnote}

%%% Change title format to be more compact
\usepackage{titling}

% Create subtitle command for use in maketitle
\newcommand{\subtitle}[1]{
  \posttitle{
    \begin{center}\large#1\end{center}
    }
}

\setlength{\droptitle}{-2em}

  \title{Korrespondenssianalyysi - graafinen ja geometrinen data-analyysin
menetelmä}
    \pretitle{\vspace{\droptitle}\centering\huge}
  \posttitle{\par}
    \author{Jussi Hirvonen}
    \preauthor{\centering\large\emph}
  \postauthor{\par}
      \predate{\centering\large\emph}
  \postdate{\par}
    \date{Versio 0.01, tulostettu 2018-08-08}


\begin{document}
\maketitle

{
\setcounter{tocdepth}{1}
\tableofcontents
}
\hypertarget{alkutoimia}{%
\chapter*{Alkutoimia}\label{alkutoimia}}
\addcontentsline{toc}{chapter}{Alkutoimia}

Ladataan r-paketit, ei tulosteta dokumenttiin. Pelkkä YAML- `front
matter', lisäkonfiguroinnit tiedostoissa \_bookdown.yml ja \_output.yml.

Dokumettiin kuuluvat Rmd-tiedostot luetellaan eksplisiittisesti (ei
vielä).

\textbf{Ideoita}

\begin{enumerate}
\def\labelenumi{\arabic{enumi}.}
\item
  Ehkä automaattista R-kirjastojen dokumentointia voisi harkita?
\item
  Saako gitbook-tulosteessa päälle asetuksen code\_folding: hide? Vaatii
  teeman (theme), jos tarpeen voi lisätä \_output.yml - tiedostoon esim.
  html\_book - formaatin.
\end{enumerate}

\hypertarget{johdanto}{%
\chapter{Johdanto}\label{johdanto}}

\textbf{xyz} Kirjoitetaan disposition pohjalta, keräillään kaikki
yleiset ca-luonnehdinnat yhteen paikkaan eli johdantoon.

\textbf{Mahdollisia lisäyksiä}

\begin{enumerate}
\def\labelenumi{\arabic{enumi}.}
\item
  Lyhyt esitys CA:n historiasta (vai omaksi luvuksi, luku 2)?
\item
  Käytetyt ohjelmistot, tekninen ympäristö ml. bookdown-asetukset. Ehkä
  paremmin omaksi liitteeksi?
\item
  Tavoitteet, sisältö, rajaukset (jota voi myöhemmin täydentää)
\item
  Muutamat puutteet, onko kerrottava tässä?
\end{enumerate}

\begin{itemize}
\item
  data: ei huomioida sitä, että otoskoot vaihtelevat aika paljon eli
  ``maapainot'' eri suuruisia
\item
  ei huomioida muitakaan otantaan liittyviä asioita (tämä ainakin
  mainittava data-osuudessa)
\item
  kuvaileva menetelmä, mutta mikä on tutkimusongelma? Sellainen pitäisi
  olla.
\end{itemize}

**zxy* Mitä on korrespondenssianalyysi? Muutamalla kappaleella. Yksi
kappale historiasta.

\hypertarget{tutkielman-tavoite-tutkimusongelma}{%
\section{Tutkielman tavoite
(tutkimusongelma?)}\label{tutkielman-tavoite-tutkimusongelma}}

\textbf{ks} Esitellään korrespondenssianalyysin käsitteet ja graafisen
analyysin periaatteet.

\textbf{ks} Tämän voi tehdä yksinkertaisen korrespondenssianalyysin
avulla. Yksinkertainen kahden luokittelumuuttujan
korrespondenssianalyysi antaa graafisen analyysin ``\ldots{}perussäännöt
tulkinnalle. Kaikki muut korrespondenssianalyysin muodot ovat saman
algoritmin soveltamista toisen tyyppiisiin datamatriiseihin, ja
tulkintaa sovelletaan vastaavasti (with the consequent adaptation of the
interpretation)''\citep[ , s.
437]{RefWorks:doc:5a857a44e4b0ed2d44664d84}

\hypertarget{data}{%
\chapter{Data}\label{data}}

\textbf{xyz} Voisi miettiä paremman otsikon

\hypertarget{yksinkertainen-korrespondenssianalyysi}{%
\chapter{Yksinkertainen
korrespondenssianalyysi}\label{yksinkertainen-korrespondenssianalyysi}}

\textbf{xyz} Tässä yksi kysymys, kuusi maata, peruskäsitteet lopussa

\hypertarget{yksinkertaisen-korrespondenssianalyysi---tulkinnan-syventaminen}{%
\chapter{Yksinkertaisen korrespondenssianalyysi - tulkinnan
syventäminen}\label{yksinkertaisen-korrespondenssianalyysi---tulkinnan-syventaminen}}

\textbf{xyz} Tarkasti läpi keskeiset tulokset ja niiden tulkinta,
kaavat, ja ytimenä eri kuvat eli kartat.

\hypertarget{yksinkertaisen-korrespondenssianalyysin-laajennuksia}{%
\chapter{Yksinkertaisen korrespondenssianalyysin
laajennuksia}\label{yksinkertaisen-korrespondenssianalyysin-laajennuksia}}

\textbf{xyz} Yksinkertainen korrespondenssianalyysi on menetelmän
tulkinnan perusta. Perusasetelmaa kahden luokittelumuuttujan
ristiintaulukoinnista voidaan laajentaa monipuolisempiin
tutkimusasetelmiin. Varsinainen useamman muuttujan
korrespondenssianalyysi (MCA - multiple correspondence analysis)
esitellään seuraavassa luvussa.

\bibliography{jhca2018.bib,packages.bib}


\end{document}
